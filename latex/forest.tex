\documentclass[oneside,a4paper]{book}%
\usepackage[final]{microtype}%
\usepackage{amsthm,mathtools}%
\usepackage{xcolor}%
\usepackage[colorlinks=true,linkcolor={blue!30!black}]{hyperref}%
\newtheorem{theorem}{Theorem}[chapter]%
\newtheorem{lemma}[theorem]{Lemma}%
\newtheorem{observation}[theorem]{Observation}%
\newtheorem{axiom}[theorem]{Axiom}%
\newtheorem{corollary}[theorem]{Corollary}%
\theoremstyle{definition}%
\newtheorem{definition}[theorem]{Definition}%
\newtheorem{construction}[theorem]{Construction}%
\newtheorem{example}[theorem]{Example}%
\newtheorem{convention}[theorem]{Convention}%
\newtheorem{exercise}{Exercise}%
\usepackage{newtxmath,newtxtext}%
\usepackage[mode=buildmissing]{standalone}%
\setcounter{tocdepth}{5}%
\setcounter{secnumdepth}{5}%
\title{Hello, World!}\author{}\begin{document}
\begin{filecontents*}[overwrite]{\jobname.bib}

\end{filecontents*}
\frontmatter\maketitle\tableofcontents\mainmatter\par{}
  Welcome to your first tree! This tree is the root of my forest.
  \begin{itemize}\item{}Build and view your forest for the first time
    \item{}Overview of the Forester markup language
    \item{}Creating new trees
    \item{}Creating your personal biographical tree
    \item{}Type theory / higher category theory projectpbn-0001\par{}Stuff for my higher category theory project
    \item{}\end{itemize}\backmatter\nocite{*}\bibliographystyle{plain}\bibliography{\jobname.bib}\end{document}